\documentclass[12pt]{article}
\usepackage{graphicx} % Required for inserting images
\usepackage{amsmath}

% Custom todo command
\newcommand{\TODO}[1]{\textcolor{red}{[#1]}}

\title{Notes for defence of thesis (2025)}
    \date{} 
\author{Tomas Reivinger}

\usepackage[parfill]{parskip} % Remove indents and add linebreaks

\begin{document}

\maketitle
\thispagestyle{empty} % Remove page number
\begin{center} This version: \today\end{center}

\section{Motivation}
% starts at p.2
Motivation for studying the effect of alcohol on acute harms, in a way, does not need very deep motivation. By lived experience, it is probably clear to most that there are harms, as well as pleasures, from consumption of alcohol.

Nevertheless, the figures from (WHO?), provides some motivating, quantitatively .... numbers..

While these numbers certainly are suggestive, there are real problems with identifying the causal effect of alcohol on health.

\medskip
\textbf{NEXT SLIDE}
\medskip

This figure shows a typical J-curve that has been reproduced many times in epidemiology. Perhaps you have heard about the benefits of drinking 1 or 2 glasses of wine? 

Such suggestions probably come from a curve like this. It shows compares the outcomes for non-drinkers (normalized to 1) with those of consume 1,2, 3 or more drinks per day. 

Higher values imply higher risk. Drinking a couple of units seems to improve health relative to non-drinkers, but more than that tends to be associated with worse outcomes.

The problem, of course, is that of omitted variable bias, reverse causality as well as other selection issues. To give one example why, the comparison group (i.e non-drinkers) contain so-called sick quitters and...

Researcher can account for that in various ways but the problems remain.

\medskip
\textbf{NEXT SLIDE} (Natural experiments a fix?)
\medskip 

As you all can imagine, using a natural experiment to solve these issues are tempting. 

TODO: List what have been done, explain that the execution is a little, or more, shaky. Perhaps the best research is... MLDA... But even this research has some real donsides. 1. no birthDAY 2. Very limited population.

Can we estimate the ATT (i,e not local ATT as in MLDA) somehow? (this is motivation for my thesis)

BE SURE TO MENTION HERE WHET I DO IN THE THESIS: i.e use a staggered adoption of systembolaget stores.


\section{Roadmap}

Briefly about the roadmap

\textbf{NEXT SLIDE}

\section{Historical background}
To understand what Systembolaget is, you need to understand why exist in the first place. 

The first thing is that Sweden had a serious drinking problem and there was a lot of effort and political capital spent on preventing excessive drinking.

By 1920, the so-called Bratt system -- a system of individual control over alcohol consumption -- was established all over Sweden. DESCRIBE WHAT IT WAS HERE, ONE SENTENCE. Even this was not, by some, considered strong enough and a referendum on a total(?) prohibition was held in 1922, with the pro-prohibitionist list by a single percentage-point.

The Brat system was ended by... in the end of 1955.



 \textbf{PLOT:} The plot shows the average consumption \emph{ converted to Liters of 100 proof alcohol}. For reference: TODO: add bottles of vodka AND important dates that you can refer to IF ASKED!

 \section{Institutional setting}
 As mentioned before, I exploit the staggered expansion of Systembolaget stores. The setting is rather simple. 

 As seen in the two plots on the right, just under half of the Swedish municipalities are treated in 1955 when Systembolaget is created.

 There were some rules, or rather guidelines that Systembolaget followed. Initially, the distance from the main city center to the nearest store needed to excced 40 kilometers and the main town had to have at least 5000 residents. These requirements were later relaxed.

 More importantly, there were alternative ways to acquire alcohol than stores. Namely, through a system of delivery points.

 \textbf{NEXT SLIDE}

 ..





\section{Data}

First, the data on all-cause mortality is a municipality level time-series that captures the universe of deaths by any cause. The municipality in the data is place of residence, not place of death. The reason for is simply that the national board of health only started collecting information on the place of death in 2015. The NBHW (socialstyrelsen) has estimated that about 80\% of cases occur in the municipality of residence.

The data over car accidents contains all accidents -- light, severe and fatal -- that \emph{have been reported by the police}  and the municipality variable represents the place where the accident occurs. In general, the more severe the accident is, the more likely it is that the police receives a report. Of course, lighter accidents that involve a drunk driver have a higher probability of being missing in the dataset.

We can return to the problems with the municipality-varible during the discussion later if necessary. I will just note that there are some problems here with measurement errors that are hard fix.

Finally, I have the data on sales -- in liters of alcohol -- by municipality that I use to estimate the 'first stage' effects.

 \textbf{NEXT SLIDE}

 Now I would like to say a few words about the municipality reforms and the sample restriction I do for the mortality data. 

 Considering that there were 900 municipalities in the beginning of the period at hand, and only about 290 in the end, you might -- rightly I might add -- wonder about the idiosyncratic use of municipality border definitions. 


Firstly, Statistics Sweden have chosen to aggregate the data using the 1995 borders so there is no way for me to even consider using the smaller (landskommuner) in the eventstudy. 

Second, even if I could use the older definitions, aggregating to a higher level would probably have been preferable anyway to get a longer time-series. 

On the other hand, there is a discussion to be had about who is in the treated group. Arguably those living near a systembolag are more likely to take up treatment. This favors the smaller municipality definitions. If I had individual-level data, I would even consider drawing a smaller circle around each store and use that as the treatment definition.

Lastly, I have restricted the data to the years 1968-1996. I have done this for mainly two reasons:

One: The estimators by Callaway and Sant'Anna (2021) and de Chaisemartin and D'Haultfœuille 2023 should be numerically equivalent without controls. They are not as you can see on the eventstudy on the right.

The other point is related to what is the most ethical point in regard to scientific praxis. We all agree that using sample selection to squeeze out an effect is unethical, but if the sample selection is "robust", then I would argue that sample selection is OK as long as we are being transparent.

Two: Because the mortality data was aggregated to the 1995 levels, that was the period I worked with in the early stages and the results you will see are robust to changing the end year to, for example 1985 or 1990.

Either way, I belive my conclusion would have been about the same, if not the same using 1996 or 2023.

Q: ANY QUESTIONS SO FAR?

 \textbf{NEXT SLIDE}

 \section{Empirical strategy and design}

The population equation of interest is

\begin{equation}
    Y_{mt} = \gamma_m + \lambda_t + \beta_{mt}  \text{store}_{mt} + \epsilon_{mt}\, ,
    \label{eq:pop-eq-of-interest}
\end{equation}

where $m = 1,2,\cdots, 115$ indicates municipality and $t = 1,2, \cdots, 29$ indicates time, in this case years. The terms $\gamma_m$ and $\lambda_t$ indicate municipality and time fixed effects, respectively. The dependent variable $Y_{mt}$ is all-cause mortality, population size and sales in liters of alcohol. An indicator $\text{store}_{mt}$ takes the value 1 if municipality $m$ at year $t$ has opened a Systembolaget store. Importantly, once a municipality is treated it stays treated. 

It has historically been common to estimate (\ref{eq:pop-eq-of-interest}) using two-way fixed effects (TWFE), but it is now well known that if there exist heterogeneous effects, dynamic or in treatment, $\beta_{mt}$ will be biased \parencite{roth2023a}.

A side note: I choose not to include the classic leads and lags equation -- perhaps influenced by my supervisor -- because Callaway-st-Anna does not really use leads and lags but creates simple did pairs using long differences and then and then aggregate them. If there are any strong objections here I will of course consider adding a traditional eventstudy equation during revision. A important thing to know is that the Callaway-st-Anna estimator uses all the data and the result is not impacted by the choice of lead and lags. 


\section{Results}


\subsection{Sales 2x store}
TODO: DESCRIBE THE TEST

The Saturday experiments by Norström and Skog tested the effect of letting Systembolaget sell alchol on Saturdays. The test was conducted twice: first in some test regions and then for the whole contry. Phase one was conducted in the beginning of the year 2000 and the second phase in mid-2001. 

Recall that the 2nd store test uses the 1978-2008 data, so there is partial overlap.



TODO: After showing Sutva, just show the HONEST PLOT and descripte it very briefly. 

 
 













\end{document}
